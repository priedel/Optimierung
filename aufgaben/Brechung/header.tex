%!TEX TS-program = pdflatex
\documentclass[numbers=noenddot,abstracton]{\scrartclScrreprt}

\usepackage[numbers]{natbib}
\usepackage[utf8x]{inputenc}
\usepackage[T1]{fontenc}
\usepackage{lmodern}
\usepackage{layout}
\setlength{\parindent}{0em}

\renewcommand{\baselinestretch}{1.2}
\renewcommand{\arraystretch}{1}

\let\oldmarginpar\marginpar
\renewcommand{\marginpar}[1]{\-\oldmarginpar[\raggedleft\scriptsize\hspace{0pt}#1]%
{\raggedright\footnotesize #1}}

%Damit \today ein Deutsch Formatiertes Datum zurueckgibt.  
\usepackage[ngerman, num, orig]{isodate}
\usepackage[ngerman]{babel}   

\monthyearsepgerman{\,}{\,} 

\usepackage{amssymb,amsmath,fancybox,graphicx,wrapfig,color,lastpage,fancyhdr,verbatim,epstopdf,a4wide}
\usepackage{paralist}	%aufzählungen in Tabelle machen

\usepackage{setspace}
\usepackage{epsfig}
\usepackage{supertabular}{\tiny }
\usepackage[font=small,labelfont=bf]{caption}
\usepackage{subcaption}
\usepackage{footnote}
\usepackage{float}
\usepackage{multirow}
\usepackage{etex}
\usepackage{pdfpages}
\usepackage{color} 
\usepackage{placeins} 
\usepackage{booktabs}


\usepackage[makeroom]{cancel}
\usepackage{array}
\usepackage{trfsigns}
\usepackage{textcomp}


% Kompaktes Itemize
\usepackage{enumitem}
\newlist{compactitemize}{itemize}{1}
\setlist[compactitemize]{label=\textbullet, nosep, leftmargin=10pt}


\usepackage{tikz}
\usetikzlibrary{shapes}
\usepackage{pgfplots}
\pgfplotsset{every axis plot/.style={line width=1pt}}
\pgfset{number format/1000 sep={}}
\pgfplotsset{tick label style={/pgf/number format/fixed}}
\pgfplotsset{scaled ticks=false}
\pgfplotsset{compat=newest}

\newcommand{\trauthorB}{Philipp Riedel}

\newcommand{\trauthorA}{Armin Stocklin}

\newcommand{\trprof}{Prof. Dr. A. Müller}
\newcommand{\trfachgebiet}{Mathematisches Seminar}
\newcommand{\trfakultaet}{Elektrotechnik}
\newcommand{\trdate}{\today}

\newcommand{\titleinfo}{Optimierung mittels Brechungsgesetz}
\newcommand{\authorinfo}{\trauthorA, \trauthorB}

\usepackage[hyphens]{url}	%URL handling und darstellung
\urlstyle{tt}

\usepackage[pdftitle={\titleinfo},
						pdfauthor={\authorinfo},
						pdfcreator={TeXStudio, LaTeX with hyperref},
						plainpages=false,
						pdfpagelabels,
						colorlinks,
						linkcolor=black,
						filecolor=black,
						citecolor=black,
						urlcolor=black]{hyperref}

\usepackage{tabularx}

\renewcommand{\captionfont}{\scriptsize\slshape}

\newcommand{\figref}[1]{Abbildung~\ref{#1}}
\newcommand{\subfigref}[2]{\figref{#1}.#2}
\renewcommand{\eqref}[1]{Gleichung~\ref{#1}}
\newcommand{\tabref}[1]{Tabelle~\ref{#1}}
\renewcommand{\pageref}[1]{Seite~\ref{#1}}
\newcommand{\chapref}[1]{Kapitel~\ref{#1}~(\nameref{#1})}
\newcommand{\secref}[1]{Abschnitt~\ref{#1}}
\newcommand{\lstref}[1]{Listing~\ref{#1}}

\newcommand{\cditem}[2]{\item[$\blacktriangleright$] \textbf{#1} \\ #2}
\newcommand{\cdrawitem}[1]{\item[$\blacktriangleright$] \textbf{#1}}
\newcommand{\cdsubitem}[2]{\item[$\vartriangleright$] \textbf{#1} \\ #2}

\newcommand{\draftmarker}[1]{\colorbox{yellow}{#1}}

\newcommand{\texttodo}[1]{\textcolor{red}{Todo: #1}}
	
\setlength{\unitlength}{1mm}

\setcounter{secnumdepth}{3}
\setcounter{tocdepth}{3}
	
%Abbildungsnumerierung anhand Kapitel
\renewcommand{\thefigure}{\arabic{section}.\arabic{figure}}
\makeatletter \@addtoreset{figure}{section} \makeatother

%%%%%%%%%%%%%%%%%%%%%%%%%%%%%%%%%%%%%%%%%%%%%%%%%%%%%%%%%%%%%%%%
% Begin define headings
%%%%%%%%%%%%%%%%%%%%%%%%%%%%%%%%%%%%%%%%%%%%%%%%%%%%%%%%%%%%%%%%
%\renewcommand{\headrulewidth}{0.4pt}
%\renewcommand{\footrulewidth}{0.4pt}

\lhead[\scriptsize\nouppercase{\leftmark}]{}%\textbf{\titleinfo}}
\chead[]{}
\rhead[]{\scriptsize\nouppercase{\leftmark}}%\textbf{\titleinfo}]{}

\lfoot[\scriptsize\thepage]{\scriptsize{}}
\cfoot[]{}
\rfoot[\scriptsize{}]{\scriptsize\thepage}

\AtBeginDocument{\numberwithin{equation}{section}} %Damit Gleichungen mit Section Nummer beginnen
\AtBeginDocument{\numberwithin{figure}{section}}
\AtBeginDocument{\numberwithin{table}{section}}

