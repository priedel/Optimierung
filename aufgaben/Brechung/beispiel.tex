\section{Rechnung an einem Konkreten Beispiels}

\subsection{Brechungsgesetz von Snellius \label{brechungsgesetz}}
\cite{Wikipedia} Aus dem Fermatschen Prinzip lässt sich das Brechungsgesetz von Snellius herleiten.
Das Licht legt den Weg vom Startpunkt $P_0$ über den Brechungspunkt $P_1$ 
nach dem Endpunkt $P_2$ zurück, siehe \figref{Ab:brechung}.
\begin{figure}[H]
	\includegraphics[width=0.8\textwidth]{./picture/Brechung.pdf}
	\caption{Skizze des Brechungsgesetzes von Snellius}
	\label{Ab:brechung}
\end{figure}

Dadurch kann die zurückgelegte Zeit berechnet werden, siehe \eqref{snelliusT}.
\begin{align}
t(x) = t_1 + t_2 = \frac{s_1}{c_1} + \frac{s_2}{c_2} = \frac{|P_1 - P_0|}{c_1} + \frac{|P_2 - P_1|}{c_2} \notag \\
= \frac{\sqrt{a^2 + x^2}}{c_1} + \frac{\sqrt{(d-x)^2 + b^2}}{c_2} \label{snelliusT}
\end{align}

Leiten wir diese Funktion nun nach $dx$ ab, finden wir eine \texttodo{die ? Extrem....} Extremastelle von $t$, siehe \eqref{snelliusDx}.
\begin{align}
	\frac{dt}{dx} = \frac{2 \cdot x}{2 \cdot c_1 \cdot \sqrt{a^2 + x^2}} + \frac{-2 \cdot (d-x)}{2 \cdot c_2 \cdot \sqrt{(d-x)^2 + b^2}} = \notag \\
	\frac{x}{c_1 \cdot \sqrt{a^2 + x^2}} - \frac{(d-x)}{c_2 \cdot \sqrt{(d-x)^2 + b^2}} = 0 \label{snelliusDx}
\end{align}

Aus \figref{Ab:brechung} ist gut ersichtlich das folgende Substitutionen \ref{substitution} durchgeführt werden können

\begin{align}
	x = \sin(\alpha) \cdot \sqrt{a^2 + x^2} \notag \\
	d-x = \sin(\beta) \cdot \sqrt{(d -x)^2 + b^2} \label{substitution}
\end{align}


\texttodo{sehr grosser sprung, aufwändig zum nachvollziehen, zwischenschritte einfügen}
daraus ergibt sich das Verhältnis gemäss \eqref{snellius}.
\begin{equation}
	0 = \frac{\sin(\alpha)}{c_1} - \frac{\sin(\beta)}{c_2} \Leftrightarrow\frac{c_2}{c_1} = \frac{\sin(\beta)}{\sin(\alpha)}
	\label{snellius}
\end{equation}


%Diese Funktion bis jetzt ist ja schön und gut, 
%jedoch konnte noch nicht nachgewiesen werden, 
%dass die Extremalstelle auch wirklich ein Minimum ist. 
%Deshalb müssen wir nun noch die zweite Ableitung berechnen.

%\[
%0 > 
%\frac{dt^2}{d^2x} \left(\frac{\sqrt{a^2 + x^2}}{c_1} + 
%\frac{\sqrt{(d-x)^2 + b^2}}{c_2}\right) =
%\frac{dt}{dx} \left(\frac{x}{c_1 \cdot \sqrt{a^2 + x^2}} - 
%\frac{(d-x)}{c_2 \cdot \sqrt{(d-x)^2 + b^2}} \right) = \phantom a
%\]
%\[
%- \frac{x^2}{c_1 \cdot \sqrt{(a^2 + x^2)}^{3}}
%+ \frac{1}{c_1 \cdot \sqrt{a^2 + x^2}}
%+ \frac{(d-x)\cdot(x-d)}{c_2 \cdot \sqrt{(b^2 + (d - x)^2)}^{3}}
%+ \frac{1}{c_2 \cdot \sqrt{b^2 + (d-x)^2}} = \phantom a
%\]
%\[
%\frac{a^2}{c_1 \cdot \sqrt{(a^2 + x^2)}^{3}}
%+ \frac{b^2}{c_2 \cdot \sqrt{(b^2 + (d - x)^2)}^{3}} = \phantom a
%\]

\subsection{Reflexionsgesetz}
\cite{Wikipedia} Auf gleiche weise wie das das Brechungsesetz aus dem Fermatschen Prinzip hergeleitet wird, 
lässt sich daraus auch das Reflexionsgesetz ableiten.
Das Licht legt den Weg vom Startpunkt $P_0$ über den Spiegelpunkt $P_1$ 
nach dem Endpunkt $P_2$ zurück. Dadurch kann die zurückgelegte Zeit berechnet werden, siehe \eqref{reflexion}.



\begin{align}
t(x) = t_1 + t_2 = \frac{s_1 + s_2}{c} = \frac{|P_1 - P_0| + |P_2 - P_1|}{c} \notag \\
= \frac{\sqrt{a^2 + x^2} + \sqrt{(d-x)^2 + b^2}}{c} \label{reflexion}
\end{align}

$t(x)$ nach $dx$ abgeleitet ergibt eine Extremalstelle  von $t$, siehe \eqref{reflexionDx}.

\begin{align}
\frac{dt}{dx} = \frac{1}{c} \cdot \frac{2 \cdot x}{2 \cdot \sqrt{a^2 + x^2}} + \frac{-2 \cdot (d-x)}{2 \cdot \sqrt{(d-x)^2 + b^2}} \notag \\
= \frac{x}{ \sqrt{a^2 + x^2}} - \frac{(d-x)}{ \sqrt{(d-x)^2 + b^2}} \label{reflexionDx}
\end{align}

\begin{figure}[H]
	\includegraphics[width=0.8\textwidth]{./picture/Spiegelung.pdf}
	\caption{Skizze des Reflexionsgesetzes}
	\label{Ab:spiegelung}
\end{figure}

Aus \figref{Ab:spiegelung} ist gut ersichtlich das die Substitution \ref{substitution}  von \secref{brechungsgesetz} auch hier nützlich ist.


\texttodo{sehr grosser sprung, aufwändig zum nachvollziehen, zwischenschritte einfügen}
daraus ergibt sich das Eintritts- und Austrittswinkel gleich sind, siehe \eqref{brechung}.


\begin{equation}
0 = \sin(\alpha) - \sin(\beta) \Leftrightarrow \sin(\beta) = \sin(\alpha) \Leftrightarrow\beta = \alpha
\end{equation}

In \figref{Ab:spiegelung} ist ersichtlich, dass die Linie des Startpunktes bis zum 
gespiegelten Endpunkt eine Gerade ist, welche der Funktion des kürzesten Weges entspricht.

%Diese Funktion bis jetzt ist ja schön und gut, 
%jedoch konnte noch nicht nachgewiesen werden, 
%dass die Extremalstelle auch wirklich ein Minimum ist. 
%Deshalb müssen wir nun noch die zweite Ableitung berechnen.

%\[
%0 > 
%\frac{dt^2}{d^2x} \left(\frac{\sqrt{a^2 + x^2}}{c_1} + 
%\frac{\sqrt{(d-x)^2 + b^2}}{c_2}\right) =
%\frac{dt}{dx} \left(\frac{x}{c_1 \cdot \sqrt{a^2 + x^2}} - 
%\frac{(d-x)}{c_2 \cdot \sqrt{(d-x)^2 + b^2}} \right) = \phantom a
%\]
%\[
%- \frac{x^2}{c_1 \cdot \sqrt{(a^2 + x^2)}^{3}}
%+ \frac{1}{c_1 \cdot \sqrt{a^2 + x^2}}
%+ \frac{(d-x)\cdot(x-d)}{c_2 \cdot \sqrt{(b^2 + (d - x)^2)}^{3}}
%+ \frac{1}{c_2 \cdot \sqrt{b^2 + (d-x)^2}} = \phantom a
%\]
%\[
%\frac{a^2}{c_1 \cdot \sqrt{(a^2 + x^2)}^{3}}
%+ \frac{b^2}{c_2 \cdot \sqrt{(b^2 + (d - x)^2)}^{3}} = \phantom a
%\]

\subsection{Fata Morgana}
\begin{figure}[H]
	\includegraphics[width=0.8\textwidth]{./picture/FataMorgana.pdf}
	\caption{Skizze der Funktion einer Fata Morgana}
	\label{Ab:fata}
\end{figure}
Wir nehmen an, wir hätten eine Fata Morgana. 
Dabei hat der Brechungsindex die Funktion nach \eqref{brechungsindexFunktion} in Abhängigkeit von $y$ 
(Der Höhe über dem Boden).

\begin{equation}
	n = n_0 (1 - \varepsilon e^{- \alpha y})
	\label{brechungsindexFunktion}
\end{equation}

\texttodo{welcher effekt ist relativ klein?}
Dabei ist der Effekt relativ klein, da $\epsilon << 1$ ist und 
die Skalenlänge $\alpha$ in der Grössenordnung von $\alpha = 1 \cdot m^{-1}$ ist.
Zusätzlich können wir für die Skalierung der x-Richtung die nach \eqref{lichtge} definierte konstante $C$ einführen.

\begin{equation}
	C = n \frac{dx}{ds}
	\label{lichtge}
\end{equation}

Wenn die Strahlengleichung komponentenweise für $r = (y(x), x)$ betrachten wird kann daraus die Strahlengleichung, \eqref{strahlengleichung}, hergeleitet werden.

\begin{equation}
\frac{d}{ds} \left ( n \frac{dy}{ds} \right ) = \frac{d}{dx} \left ( n \frac{dy}{dx} \frac{dx}{ds} \right ) \frac{dx}{ds} =
\frac{d}{dx} \left ( C \frac{dy}{dx} \right ) \frac{C}{n} = \frac{\partial n(y)}{\partial y}
\label{strahlengleichung}
\end{equation}

Da die Konstante $C$ nur die x-Achse skaliert, wird diese auf den Wert $1$ festgelegt.
Wegen 
\texttodo{stimmt diese gleichung? $2 n \partial n / \partial y = \partial n^2 / \partial y$ meiner meinung nach geht da faktor 2 verloren, ist aber schon spät in der Nacht}
\begin{equation}
	2 n \partial n / \partial y = \partial n^2 / \partial y \qquad \text{und} \qquad n^2 \simeq n_0^2(1 - 2 \varepsilon e^{-\alpha y}) \notag
\end{equation}erhalten wir \eqref{difGleich}.

\begin{equation}
	\frac{d^2 y(x)}{dx^2} = \frac{1}{2} \frac{\partial}{\partial y} n^2(y) = n_0^2 \varepsilon \alpha e^{-\alpha y}
	\label{difGleich}
\end{equation}

Diese Gleichung kann mit elementaren Methoden gelöst werden. 
Zudem wird der Steigungswinkel $\varphi$ eingeführt daraus ergibt sich
$\kappa = (\frac{\alpha}{2}) \tan(\varphi)$ daraus ergibt sich die übersichtlichere \eqref{fataFunktion}.
\texttodo{verständlich beschreiben mit zwischen schritten damit sich die diffgleichung und die funktion die gefunden wird ersichtlich sind}
\begin{equation}
	y = y_0 + \frac{1}{\alpha} ln(\cosh^2(\kappa(x - x_0))) \xrightarrow{\kappa (x - x_0 ) >> 1} y =	y_0 + \frac{2 \kappa}{\alpha} (x - x_0)
	\label{fataFunktion}
\end{equation}


In grossen Abständen zum Spiegelpunkt $x_0$ ist die Ausbreitung des Lichtes geradlinig.
Der Beobachter sieht, wie erwarted zwei Bilder, wobei eines auf dem Kopf steht.

\subsection{Lichtwellenleiter}

Hier kommt das vierte Beispiel.