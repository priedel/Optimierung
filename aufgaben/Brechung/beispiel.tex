\section{Rechnung an einem Konkreten Beispiels}

\subsection{Brechungsgesetz von Snellius}
\cite{Snellius}Aus dem Fermatschen Prinzip lässt sich das Brechungsgesetz von Snellius herleiten.
Das Licht legt den Weg vom Startpunkt $P_0$ über den Brechungspunkt $P_1$ 
nach dem Endpunkt $P_2$ zurück. Dadurch kann die zurückgelegte Zeit berechnet werden.


\fbox{
 \begin{picture}(85,80)
  \put(0,40){\line(1,0){85}}
  \put(0,80){\line(3,-2){60}}
  \put(60,40){\line(2,-3){26}}
  \put(0,60){a}
  \put(0,20){b}
  \put(43,0){d}
 \end{picture}
}

\[
t(x) =
t_1 + t_2 =
\frac{s_1}{c_1} + \frac{s_2}{c_2} =
\frac{|P_1 - P_0|}{c_1} + \frac{|P_2 - P_1|}{c_2} =
\frac{\sqrt{a^2 + x^2}}{c_1} + \frac{\sqrt{(d-x)^2 + b^2}}{c_2}
\]

Leiten wir diese Funktion nun nach $x$ ab, finden wir eine Extremalstelle von $t$.

\[
0 = 
\frac{dt}{dx} =
\frac{2 \cdot x}{c_1 \cdot 2 \cdot \sqrt{a^2 + x^2}} + 
\frac{-2 \cdot (d-x)}{c_2 \cdot 2 \cdot \sqrt{(d-x)^2 + b^2}} =
\frac{x}{c_1 \cdot \sqrt{a^2 + x^2}} - 
\frac{(d-x)}{c_2 \cdot \sqrt{(d-x)^2 + b^2}}
\]

Aus dem Bild kann man schön sehen, dass $x = \sin(\alpha) \cdot \sqrt{a^2 + x^2}$
und $d-x = \sin(\beta) \cdot \sqrt{(d -x)^2 + b^2}$

\[
0 = 
\frac{\sin(\alpha)}{c_1} - \frac{\sin(\beta)}{c_2} \Leftrightarrow
\frac{c_1}{c_2} = \frac{\sin(\beta)}{\sin(\alpha)}
\]

Diese Funktion bis jetzt ist ja schön und gut, 
jedoch konnte noch nicht nachgewiesen werden, 
dass die Extremalstelle auch wirklich ein Minimum ist. 
Deshalb müssen wir nun noch die zweite Ableitung berechnen.

\[
0 > 
\frac{dt^2}{d^2x} \left(\frac{\sqrt{a^2 + x^2}}{c_1} + 
\frac{\sqrt{(d-x)^2 + b^2}}{c_2}\right) =
\frac{dt}{dx} \left(\frac{x}{c_1 \cdot \sqrt{a^2 + x^2}} - 
\frac{(d-x)}{c_2 \cdot \sqrt{(d-x)^2 + b^2}} \right) = \phantom a
\]
\[
- \frac{x^2}{c_1 \cdot \sqrt{(a^2 + x^2)}^{3}}
+ \frac{1}{c_1 \cdot \sqrt{a^2 + x^2}}
+ \frac{(d-x)\cdot(x-d)}{c_2 \cdot \sqrt{(b^2 + (d - x)^2)}^{3}}
+ \frac{1}{c_2 \cdot \sqrt{b^2 + (d-x)^2}} = \phantom a
\]
\[
\frac{a^2}{c_1 \cdot \sqrt{(a^2 + x^2)}^{3}}
+ \frac{b^2}{c_2 \cdot \sqrt{(b^2 + (d - x)^2)}^{3}} = \phantom a
\]

\subsection{Spiegelgesetz}
Hier kommt das zweite Beispiel.

\subsection{Fata Morgana}
Hier kommt das dritte Beispiel.

\subsection{Lichtwellenleiter}

Hier kommt das vierte Beispiel.