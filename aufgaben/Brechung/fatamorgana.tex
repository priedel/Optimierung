\section{Berechnungen mittels Euler-Lagrange Gleichung}
\subsection{Einleitung}
Einige Optimierungsprobleme lassen sich direkt lösen, indem sie in die hergeleitete Euler-Lagrange-Gleichung eingesetzt werden können. Andere Optimierungsprobleme, bei denen eine optimale Funktion gesucht wird, können gelöst werden indem sie das Prinzip der Variation der Euler-Lagrange-Gleichung benutzen. Dazu reicht jedoch nicht mehr ein einfaches einsetzen, sondern es muss eine neue Gleichung hergeleitet werden. Die gesuchte Funktion wird dann gefunden, indem für $g(y)$ und $g'(y)$ konkret $y(x)$ und $y'(x)$ eingesetzt wird. Aber dazu später noch mehr.
\subsection{Motivation für die Variationsrechnung}
In \secref{brechungsgesetz} werden einfach geometrische zusammenhänge hergeleitet wenn zwei verschiedene benachbarte Medien einen unterschiedlichen Brechungsindex haben. komplexere Probleme können mit diesen Überlegungen nicht mehr gelöst werden. Der Unterschied liegt darin, dass der Brechungsindex sich kontinuierlich im Raum ändert. In   einem  beliebigen  Medium  ist  der Brechungsindex variabel, eine Funktion $n(x,y)$. Das  Problem wird erst mal nur 2-Dimensional betrachtet. Mit diesen Begebenheiten ist es nicht mehr möglich, eine einfache geometrische Überlegung  zur  Bestimmung  des  Strahles  zu verwenden. der Lichtstrahl wird im allgemeinen gekrümmt sein.
\subsection{Beschreibung einer Fata Morgana}
Bei der physikalischen Betrachtung der Lichtausbreitung wird davon ausgegangen, dass sich Licht in
geradlinig verlaufenden Strahlen fortpflanzt. Das Auge erwartet das Objekt, von dem die Lichtstrahlen
kommen, in rückwärtiger geradliniger Verlängerung der Richtung, welche das Licht beim eintreffen in das Auge besitzt.
Wenn das Licht ein Medium, z.B. mit unterschiedlichen Brechungsindizes durchquert, ändert es seine Richtung, siehe \secref{brechungsgesetz}. Dies erklärt z.B. die verkürzten Beinen im Schwimmbad.
Eine kompliziertere Situation liegt vor, wenn der Brechungsindex des durch-strahlten Mediums kontinuierlich variiert. 
Dies ist der Fall, wenn die Luft in der Nähe eines stark aufgeheizten Untergrundes erwärmt
wird und infolge der dadurch bewirkten Dichteänderung einen räumlichen variierenden Brechungsindex annimmt. 
Das Licht ändert stetig seine Richtung, dass führt zu Phänomenen wie Luftspiegelungen von Autolichtern oder einer Fata Morgana, an heissen Tagen \cite{fataEinleitung}.
In \figref{Ab:fataEinleitung} wird die Wahrnehmung des Auges und eine Richtungsänderung des Lichtes gezeigt.
\begin{figure}[H]
\begin{center}
\includegraphics[width=0.55\textwidth]{./picture/FataEinleitung.png}
	\caption{Das Auge interpretiert, dass die Lichtstrahlen mit veränderter Richtung aus der tangentialen Verlängerung kommen und sieht dort das Bild}
	\label{Ab:fataEinleitung}
\end{center}	
\end{figure}
\subsection{Untersuchung der Krümmungs-Eigenschaften bei einer räumlichen Dichteänderung \label{sec:Krümmung}}
Bei einer Fata Morgana hat die Luft eine optische Dichte von $n(x,y)$. 
Dabei bewegt sich das Licht mit der Geschwindigkeit $c/n(x,y)$. 
Die benötigte Zeit, damit das Licht vom Punkt $(x_0, y_0)$ nach $(x_1, y_1)$ braucht,
kann mit dem Kurvenintegral berechnet werden (\eqref{funktionTy}).
\begin{equation}
	T(y) = \int \limits_{x_0}^{x_1} \frac{n(x,y)}{c} \sqrt{1 + y'(x)^2} dx
	\label{funktionTy}
\end{equation}
Wie es bei einem heissen Tag auf einer Strasse oder in einer Wüste zutrifft,
nehmen wir zuerst nur einmal an, dass die optische Dicht nur von $y$ abhängt und mit zunehmenden $y$ zunimmt,
$n(x,y) \rightarrow n(y)=g(y)$.
Durch die geringe Krümmung der Erde kann die $x$ Richtung vernachlässigt werden.
Dies bedeutet für die Funktion $g(y)$, dass $g(y) > 0$ und $g'(y) > 0 $ ist.
Um dieses Minimalproblem zu lösen, wird die allgemeine Euler-Lagrange-Gleichung \chapref{chapter-variationsrechnung} \eqref{lagrange-integral} mit Nebenbedingungen aufgestellt (\eqref{krümmung}).
\begin{equation}
	T(y) = \int \limits_{x_0}^{x_1} \frac{g(y)}{c} \sqrt{1 + y'(x)^2} dx = \frac{1}{c} \int \limits_{x_0}^{x_1} g(y) \sqrt{1 + y'(x)^2} dx
	\label{krümmung}
\end{equation}
Der Faktor $\frac{1}{c}$ hat keinen Einfluss auf das Integral und somit auch nicht auf das Variationsproblem, deshalb kann er weggelassen werden, um ihn nicht ständig mitschleppen zu müssen.
Es werden nun die Ableitungen der Funktion \ref{funktionRech} gebraucht. Sie sind in \ref{ableitungenLag} aufgeführt.
\begin{align}
	F(x,y,y') &= g(y) \sqrt{1 + y'^2} \label{funktionRech} \\
	\frac{\partial F}{\partial y} &= g'(y) \sqrt{1 + y'^2} \notag \\
	\frac{\partial F}{\partial y'} &= g(y) \frac{y'}{\sqrt{1 + y'^2}} \label{ableitungenLag}
\end{align}
Für die Euler-Lagrange-Gleichung muss man im zweiten Ausdruck für $y$ und $y'$ die Funktionen $y(x)$ 
sowie $y'(x)$ einsetzen und nach $\frac{d}{dx}$ ableiten, (\eqref{lagrange1}).
\begin{align}
	\frac{d}{dx} \frac{\partial F}{\partial y'} &= \frac{d}{dx} (g(y(x)) \frac{y'(x)}{\sqrt{1 + y'(x)^2}}) \notag \\ 
	&= g'(y(x)) \frac{y'(x)^2}{\sqrt{1 + y'(x)^2}} + g(y(x)) \frac{y''(x)}{\sqrt{1 + y'(x)^2}}
	 - g'(y(x)) \frac{y'(x)^2 y''(x)}{(1 + y'(x))^\frac{3}{2}} 
	 \label{lagrange1}
\end{align}
Auch im zweiten Term der Euler-Lagrange-Gleichung wird für $y'$, $y'(x)$ eingesetzt. Es ergibt sich \eqref{lagrange2}
\begin{align}
	0 &= \frac{d}{dx} \frac{\partial F}{\partial y'} - \frac{\partial F}{\partial y} \label{lagrangePrinzip} \\
		0 &= \frac{d}{dx} \frac{\partial F}{\partial y'} - \frac{\partial F}{\partial y}
	= g'(y(x)) \frac{y'(x)^2}{\sqrt{1 + y'(x)^2}} + g(y(x)) \frac{y''(x)}{\sqrt{1 + y'(x)^2}} \notag \\
	&- g'(y(x)) \frac{y'(x)^2 y''(x)}{(1 + y'(x)^2)^\frac{3}{2}}  - g'(y(x)) \sqrt{1 + y'(x)^2}
	\label{lagrange2}
\end{align}
Da $\sqrt{1 + y'(x)^2} > 0$ und somit nicht gleich Null is,t kann die Gleichung mit $\sqrt{1 + y'(x)^2}$  multipliziert werden um einen einfacheren Ausdruck zu bekommen, (\eqref{lagrange3}).
\begin{align}
	0 = g'(y(x)) y'(x)^2 + g(y(x)) y''(x) - g'(y(x)) \frac{y'(x)^2 y''(x)}{1 + y'(x)^2} - g'(y(x)) (1 + y'(x)^2)
	\label{lagrange3}
\end{align}
Auch $g(y(x))$ ist nicht gleich Null, da nach eingangs erfolgter Definition $g(y(x)) > 0$, eine Division mit $g(y(x))$ ist somit zulässig, (\eqref{lagrange4}).
\begin{align}
	\frac{g'(y(x))}{g(y(x))} (1 + y'(x)^2) - \frac{g'(y(x))}{g(y(x))} y'(x)^2 &=  y''(x) - \frac{y'(x)^2 y''(x)}{1 + y'(x)^2} \notag \\
	\frac{g'(y(x))}{g(y(x))} (1 + y'(x)^2 - y'(x)^2) &= \frac{\left(1 + y'(x)^2 \right)y''(x) - y'(x)^2 y''(x)}{1 + y'(x)^2} \notag \\
	\frac{g'(y(x))}{g(y(x))} &= \frac{y''(x) + y'(x)^2 y''(x) - y'(x)^2 y''(x)}{1 + y'(x)^2} = \frac{y''(x)}{1 + y'(x)^2}\notag \\
	\frac{g'(y(x))}{g(y(x))} &= \frac{y''(x)}{1 + y'(x)^2}
	\label{lagrange4}
\end{align}
\eqref{lagrange4} kann jetzt auf ihre Eigenschaften bezüglich der Krümmung untersucht werden. Gemäss den Definitionen von ganz am Anfang ist die linke Seite positiv. 
Der Nenner der rechten Seite ist ebenfalls positiv. Daraus resultiert, dass der Zähler der rechten Seite ebenfalls positiv sein muss (\eqref{krümmungAuswertung}).
\begin{align}
	\frac{g'(y(x))}{g(y(x))} > 0 \qquad \wedge \qquad (1 + y'(x)^2) > 0 \qquad \Rightarrow \qquad y''(x) > 0
	\label{krümmungAuswertung}
\end{align}
Da die zweite Ableitung positiv ist, muss die Funktion des Lichtstrahles konvex sein.
Dies bedeutet, dass der Lichtstrahl sich über dem Boden nach oben biegt, ganz egal wie die Funktion von $g(x)$ aussieht, solange sie monoton steigend ist.
Daraus kann die verallgemeinerte Schlussfolgerung gezogen werden, dass der Lichtstrahl konkav sein muss, wenn die Funktion $g(x)$ monoton fallend ist.
\begin{satz}
Ein Lichtstrahl wird in einem inhomogenen Medium in Richtung höherer Dichte gekrümmt, verhält sich die Dichteänderung monoton fallend oder steigend, nimmt der Lichtstrahl eine konvexe oder konkave Kurvenform an.
\index{Krümmung von Lichtstrahlen}
\end{satz}
\subsubsection{Technische Anwendung dieser Erkenntnis}
\begin{figure}[H]
\begin{center}
\includegraphics[width=0.5\textwidth]{./picture/Lichtwellenleiter.pdf}
	\caption{(a) Lichtwellenleiter mit einer Sprunghaften Änderung der optischen Dichte erzeugt einen am Übergang Reflektierenden Lichtweg. 
	(b) Lichtwellenleiter mit einer stetig ändernden optischer Dichte erzeugt einen krümmenden Lichtstrahl \cite{opticFibre}. }
	\label{lichtleiter}
\end{center}	
\end{figure}
In einem Lichtwellenleiter wird dieses Phänomen ausgenutzt, 
indem  die  optische  Dichte  eines  zylindrischen
Lichtwellenleiters  von  der Achse zum Mantel hin abnimmt.
So werden die Lichtstrahlen immer vom Mantel weggekrümmt 
und das Licht  bleibt im Leiter gefangen (\figref{lichtleiter}).
\subsection{Berechnung der Differentialgleichung einer Fata Morgana \label{sec:diffgleichung}}
Für die gesuchte Differentialgleichung der Fata Morgana nehmen wir wie bei der Krümmung an, dass die optische Dichtung nur von $y$ abhängt. 
$n(y) = g(y)$ und mit zunehmenden $y$ zunimmt. Dies bedeutet wie für die Funktion $g(y)$ wie in \secref{sec:Krümmung}, dass $g(y) > 0$ und $g'(y) > 0 $ ist.
Die Euler-Lagrange-Gleichung wird gleich wie in \secref{sec:Krümmung} berechnet. Bei \eqref{lagrange4} wird weiter gerechnet. In \eqref{diffgleichung1} werden alle Terme auf die linke Seite genommen.
\begin{align}
	g(y(x) y''(x)-g'(y(x) y'(x)^2 - g'(y(x)) =0 
	\label{diffgleichung1}
\end{align}
Dabei ist $g(y(x))$ die gesuchte Funktion des Brechungsindex $n(y)$. Mit dieser Substitution ergibt sich die Differentialgleichung
\ref{diffgleichung2}.
\begin{align}
	n y''-n' (y')^2 - n' =0 
	\label{diffgleichung2}
\end{align}
In \secref{sec:Krümmung} war anfangs definiert worden das die Dichte ab dem Boden in Richtung $y$ zunimmt weil die Temperatur über dem Boden von dem heissen Untergrund mehr erhitzt ist als in höheren Lagen. Wird jetzt diese Temperaturabnahme als linear angenommen, gilt \eqref{einsetzGl1} und deren Ableitung \eqref{einsetzGl2}.
\begin{align}
	n(y)&=my+b \label{einsetzGl1} \\
	n'(y)&=m \label{einsetzGl2}
\end{align}
Um die Differentiation Gleichung zu lösen, kann für $y$ eine hyperbolische Funktion \cite{cosh} gewählt werden (\eqref{hyperFunk}).
\begin{align}
	y&=A\cosh(Bx+C)+D \label{hyperFunk} \\
	y'&= A B sinh(B x + C) \label{hyperFunkDx} \\
	y''&= A B^2 cosh(B x + C) \label{hyperFunkD2x}
\end{align}

\subsubsection{Eine Beispiel Aufgabe}
Folgende Annahmen gelten:
\begin{align}
	y_0&=y_1=2 m & \text{Augenhöhe und Startpunkt} \notag \\
	-x_0&=x_1 = 200m & \text{Distanz zum Nullpunkt des Kordinatensystems} \notag \\
	T_A&=333.15 K & \text{Temperatur des Aspalt}  \notag \\
	T_L&=293.15 K & \text{Temperatur Luft auf Augenhöhe}  \notag
\end{align}
Daraus ergeben sich mit \eqref{brechTemp} eine Funktion für den Brechungsindex von Luft in abhängigkeit von Temperatur.
\begin{align}
	n=1+0.000293\frac{T_0}{T}
	\label{brechTemp}
\end{align}
Daraus ergeben sich die werte in \eqref{nParam} für unsere lineare gleichung $n(y)=my+b$
\begin{align}
	y&=y_0=y_1 \notag \\
	b&=n_A=1+0.000293\frac{273.15 K}{333.15 K} = 1.00024 \notag \\
	n_L&=1+0.000293\frac{273.15 K}{293.15 K} = 1.00027 \notag \\
	m&=\frac{\delta n}{y}=\frac{n_L-n_A}{y}=1.64*10^{-5} \notag \\
	n(y)&=my+b=1.64*10^{-5}y +1.00024
	\label{nParam}
\end{align}
Als erstes die allgemeine Form mit einsetzen von $n,n',y',y''$ (\eqref{HauptGl}).
\begin{align}
	(my_0+b)A B^2 cosh(B x_0 + C)-m (A B sinh(B x_0 + C))^2-m &= 0\notag \\
	(my_1+b)A B^2 cosh(B x_1 + C)-m (A B sinh(B x_1 + C))^2-m &=0 \label{HauptGl}
\end{align}
Da wir mit $-x_0=x_1$ eine Achsenspieglung haben und die Cosinus Hyperbolicus Funktion eine gerade Funktion ist, lässt sich (\eqref{HauptGl}) vereinfachen (\eqref{glvereinfacht}). Dabei wird $C$ weggelassen.
\begin{align}
	(my_0+b)A B^2 cosh(B x_0 )-m (A B sinh(B x_0 ))^2-m &= 0\notag \\
	(my_1+b)A B^2 cosh(B x_1 )-m (A B sinh(B x_1 ))^2-m &=0 \label{glvereinfacht}
\end{align}

 \eqref{hyperFunk}  ist nicht linear, deshalb muss ab hier mit numerischen Methoden weitergearbeitet werden.
 
 \subsection{Schlussfolgerungen}
 Die Erkenntnis das für $y$ eine Hyperbolische Funktion resultiert und das eine Differentialgleichung für beliebige Funktionen für $n$ eingesetzt werden können ist beachtlich. Die Variationsrechnung ist sehr mächtig, es können allgemeine Eigenschaften die nicht Funktionsabhängig sind abgeleitet werden (\secref{sec:Krümmung}). Es können auch nicht triviale Differentialgleichungen hergeleitet werden (\secref{sec:diffgleichung}). Wenn ein Problem nicht direkt mit der Euler-Lagrange-Gleichung gelöst werden kann, kann das Prinzip des etwas ''verwackeln'' und der Annahme, dass sich das Integral nicht ändert trotzdem angewendet werden. Es muss dann eine neue Gleichung hergeleitet werden welche danach wieder wie bei der Euler-Lagrange-Gleichung nach $\frac{d}{dx}$ (\eqref{lagrangePrinzip}) abgeleitet wird.