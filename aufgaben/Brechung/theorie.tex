\section{Physikalische Theorie}

\subsection{Fermatsches Prinzip}
Im Jahre 1660 fand Pierre de Fermat, 
ein Französischer Mathematiker das, 
nach ihm benannte, Fermatsche Prinzip heraus. 
Dieses lautet wie folgt:

%\begin{definition}
Der Weg, den das Licht nimmt, 
um von einem Punkt zu einem anderen zu gelangen, 
ist stets so, dass die benötigte Zeit minimal ist.
%\index{Fermatsches Prinzip 1}
%\end{definition}

Mathematisch lautet das Fermatsche Prinzip bei einem Brechungsindex $n$, 
der Lichtgeschwindigkeit $c$ und dem Weg $s$: 
Die Zeit

\[
t= \int_{s_1}^{s_2} \frac{n(s)}{c} ds =
\frac{1}{c} \int_{s_1}^{s_2} n(s) ds
\]

muss minimal sein.

Wenn die Zeit minimal ist, 
wird bei kleinen Abweichungen vom Weg die Zeit nicht viel grösser. 
Deshalb kann die oben stehende Definition auch wie folgt geschrieben werden.

%\begin{definition}
Der Weg, den das Licht nimmt, 
um von einem Punkt zu einem anderen zu gelangen, 
ist stets so, dass die Zeit, die das Licht benötigt, 
invariant gegen kleine Änderungen des Weges ist.
%\index{Fermatsches Prinzip 2}
%\end{definition}

\subsection{Formulierung mit dem optischen Weg}
Der optische Weg $L$  eines Pfades $s$ kann in einem homogenen Material 
mit dem Brechungsindex $n$ als Produkt $L = n \cdot s$ berechnet werden.
Wenn nun kein homogenes Material vorliegt und die Formel allgemein 
formuliert werden muss, so ist der optische Pfad $L = \int_{s_1}^{s_2} n(\vec x) \,\mathrm d \vec x$
Da der zurückgelegte Weg, des Lichtstrahles direkt propotional zur Zeit $s = c \cdot t$ ist,
kann das Fermatsche Gesetz umformuliert werden.

%\begin{definition}
Der Weg, den das Licht nimmt, 
um von einem Punkt zu einem anderen zu gelangen, 
ist stets so, dass der optische Weg 
\[
L(S) = \int_{s_1}^{s_2} n(\vec x) \,\mathrm d\vec x
\]
minimal ist.
%\index{Fermatsches Prinzip 2}
%\end{definition}

\subsection{Quantenmechanische Erklärung}
