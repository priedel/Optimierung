\section{Einleitung}
Wie in den vorangegangen Kapitel oft gesehen stehen für Optimierungsprobleme verschiedene Programme wie z.B. der Symplex Algorithmus zur Verfügung. Dabei wird ein Maxima oder Minima gesucht. Bei der Variationsrechnung wird, wie in \chapref{chapter-variationsrechnung} beschrieben, eine Funktion gesucht welche das Optimierungsproblem löst. Für die Variationsrechnung steht anstelle eines mathematischen Programms die Euler-Lagrange-Gleichung zur Verfügung. Wie nützlich und mächtig sie ist soll anhand von Beispielen in folgenden Seiten etwas aufgezeigt werden.

\section{Einleitende Theorie, Fermatsches Prinzip}

Im Jahre 1660 fand Pierre de Fermat, 
ein Französischer Mathematiker, das 
nach ihm benannte, Fermatsche Prinzip heraus. 
Dieses lautet wie folgt\cite{Definition}:

\begin{definition}
	Der Weg, den das Licht nimmt, 
	um von einem Punkt zu einem anderen zu gelangen, 
	ist stets so, dass die benötigte Zeit minimal ist.
	\index{Fermatsches Prinzip 1}
\end{definition}

Die \eqref{fermat} zeigt das Fermatsche Prinzip bei einem Brechungsindex $n$, 
der Lichtgeschwindigkeit $c$, dem Weg $s$ und dem Ort $r$ Die Zeit muss minimal sein.

\begin{equation}
	t= \int\limits_{s_1}^{s_2} \frac{n(r)}{c} ds = \frac{1}{c} \int\limits_{s_1}^{s_2} n(r) ds
	\label{fermat}
\end{equation}


Wenn die Zeit minimal ist, wird bei kleinen Abweichungen vom Weg die Zeit nicht viel grösser. 
Deshalb kann die oben stehende Definition auch wie folgt geschrieben werden\cite{Definition}.

\begin{definition}
	Der Weg, den das Licht nimmt, 
	um von einem Punkt zu einem anderen zu gelangen, 
	ist stets so, dass die Zeit, die das Licht benötigt, 
	invariant gegen kleine Änderungen des Weges ist.
	\index{Fermatsches Prinzip 2}
\end{definition}

\subsection{Formulierung mit dem optischen Weg}
Der optische Weg $L$  eines Pfades $s$ kann in einem homogenen Material 
mit dem Brechungsindex $n$ als Produkt von $n$ und $s$ berechnet werden.
Bei einem inhomogenen Material ergibt sich für den optische Pfad die allgemeine \eqref{optischweg}.
\begin{equation}
	L(s) = \int\limits_{s_1}^{s_2} n(r) \,\mathrm d\vec s 
	\label{optischweg}
\end{equation}

Da der zurückgelegte Weg, des Lichtstrahles direkt proportional zur Zeit ist, $s = c \cdot t$,
und das Licht sich in optisch dichteren Materialien langsamer bewegt,
kann das Fermatsche Gesetz umformuliert werden\cite{Definition}. 


\begin{definition}
Der Weg, den das Licht nimmt, 
um von einem Punkt zu einem anderen zu gelangen, 
ist stets so, dass der optische Weg minimal ist.
\index{Fermatsches Prinzip 2}
\end{definition}

\subsection{Eikonalgleichung}
\texttodo{ist nicht aus der Eikongleichzung das Fermtische Prinzip abgeleitet?}
Aus dem Fermatschen Prinzip lässt sich die Eikonalgleichung herleiten.
Dabei liefert jedes Weglängenstück $ds$ folgenden Beitrag an den optischen Weg $dL$. 
Dabei ergibt sich \eqref{eikonalHerleitung}.

\begin{equation}
	dL = n \cdot ds = n e_t \cdot dr
	\label{eikonalHerleitung}
\end{equation}

Dabei ist der tangentielle Einheitsvektor der Trajektorie $e_t$ nach  \eqref{et} definiert.

\begin{equation}
	e_t = \partial r / ds
	\label{et}
\end{equation}

Die optische Wegstücke $dL$ können auch wie folgt \texttodo{was definiert werden} definiert werden, 
und es ergeben sich folgende Gleichungen.

\begin{align}
	dL = \nabla L \cdot dr \notag \\
	n e_t = n \frac{dr}{ds} = \nabla L \notag \\
	n^2 = (\nabla L)^2 \label{eikongleichung}
\end{align}

\eqref{eikongleichung} nennt sich Eikonalgleichung.
Durch Ableitung der Eikonalgleichung ergibt sich \eqref{strahlen} die wichtige Strahlengleichung der Optik.

\begin{equation}
	\frac{d}{ds} \left ( n \frac{dr}{ds} \right ) = \nabla n
	\label{strahlen}
\end{equation}

\subsection{Quantenmechanische Erklärung}
Viele Leute haben ein Problem mit dem Fermatschen Prinzip. 
Der Grund ist ganz einfach. 
Bei der ``klassischen'' Strahlenoptik geht das Licht einen bestimmten Weg, 
sieht einen Spiegel oder eine Oberfläche, bricht sich dort nach einem 
physikalischen Gesetz und geht dann weiter.
Der philosophische Ansatz vom Fermatschen Prinzip ist anders. 
Es wird davon ausgegangen, dass das Licht, bevor es einen Weg einschlägt, 
weiss wohin es muss. Es Weiss wo ein Spiegel oder eine Oberfläche kommt und 
nimmt dem entsprechend von Anfang an den Weg mit der kürzesten Zeit.
Aber woher weiss das Licht, dass sein eingeschlagener Weg der korrekte ist?
Kann man davon ausgehen, dass es die nähere Umgebung anschaut und diese Vergleicht?
Ja, genau davon kann man ausgehen. Eine Quantenmechanische Erklärung für das Fermatsche Prinzip ist, 
dass das Licht alle Wege nimmt. Jedoch löschen sich alle anderen Wege aus, 
ausser der Weg mit der kürzester Zeit wird nicht ausgelöscht.
Für die genauere Erklärung siehe \cite{quanten}.
