\section{Physikalische Theorie}

\subsection{Fermatsches Prinzip}
Im Jahre 1660 fand Pierre de Fermat, 
ein Französischer Mathematiker das, 
nach ihm benannte, Fermatsche Prinzip heraus. 
Dieses lautet wie folgt:

%\begin{definition}
Der Weg, den das Licht nimmt, 
um von einem Punkt zu einem anderen zu gelangen, 
ist stets so, dass die benötigte Zeit minimal ist.
%\index{Fermatsches Prinzip 1}
%\end{definition}

Mathematisch lautet das Fermatsche Prinzip bei einem Brechungsindex $n$, 
der Lichtgeschwindigkeit $c$, dem Weg $s$ und dem Ort $r$: 

Die Zeit
\[
t= \int_{s_1}^{s_2} \frac{n(r)}{c} ds =
\frac{1}{c} \int_{s_1}^{s_2} n(r) ds
\]
muss minimal sein.

Wenn die Zeit minimal ist, 
wird bei kleinen Abweichungen vom Weg die Zeit nicht viel grösser. 
Deshalb kann die oben stehende Definition auch wie folgt geschrieben werden.

%\begin{definition}
Der Weg, den das Licht nimmt, 
um von einem Punkt zu einem anderen zu gelangen, 
ist stets so, dass die Zeit, die das Licht benötigt, 
invariant gegen kleine Änderungen des Weges ist.
%\index{Fermatsches Prinzip 2}
%\end{definition}

\subsection{Formulierung mit dem optischen Weg}
Der optische Weg $L$  eines Pfades $s$ kann in einem homogenen Material 
mit dem Brechungsindex $n$ als Produkt $L = n \cdot s$ berechnet werden.
Wenn nun kein homogenes Material vorliegt und die Formel allgemein 
formuliert werden muss, so ist der optische Pfad $L = \int_{s_1}^{s_2} n(r) \,\mathrm d \vec s$
Da der zurückgelegte Weg, des Lichtstrahles direkt propotional zur Zeit $s = c \cdot t$ ist,
kann das Fermatsche Gesetz umformuliert werden.

%\begin{definition}
Der Weg, den das Licht nimmt, 
um von einem Punkt zu einem anderen zu gelangen, 
ist stets so, dass der optische Weg 
\[
L(s) = \int_{s_1}^{s_2} n(r) \,\mathrm d\vec s
\]
minimal ist.
%\index{Fermatsches Prinzip 2}
%\end{definition}

\subsection{Eikonalgleichung}

Aus dem Fermatschen Prinzip lässt sich die Eikonalgleichung herleiten.
Dabei ist der liefert jede Weglängenstück $ds$ folgenden Beitrag an den optischen Weg $dL$. 
Es ergibt sich folgende Formel:

\[
dL = n \cdot ds = n e_t \cdot dr
\]

Dabei ist der tangentielle Einheitsvektor der Trajektorie $e_t$ wie folgt definiert.

\[
e_t = \partial r / ds
\]

Die optische Wegstücke $dL$ können auch wie folgt definiert werden, 
und es ergeben sich folgende Gleichungen.

\[
dL = \nabla L \cdot dr
\]

\[
n e_t = n \frac{dr}{ds} = \nabla L
\]

\[
n^2 = (\nabla L)^2
\]

Die letzte Gleichung nennt sich Eikonalgleichung.
Differenzieren wir nun diese Eikonalgleichung, 
dann erhalten noch die wichtige Strahlengleichung der Optik.

\[
\frac{d}{ds} \left ( n \frac{dr}{ds} \right ) = \nabla n
\]

\subsection{Quantenmechanische Erklärung}
Viele Leute haben ein Problem mit dem Fermatschen Prinzip. 
Der Grund ist ganz einfach. 
Bei der ``klassischen'' Strahlenoptik geht das Licht einen bestimmten Weg, 
sieht einen Spiegel oder eine Oberfläche, bricht sich dort nach einem 
physikalischen Gesetz und geht dann weiter.
Der philosophische Ansatz voim Fermatschen Prinzip ist anders. 
Es wird davon ausgegangen, 
dass das Licht, bevor es einen Weg einschlägt, 
weiss wohin es muss und wo ein Spiegel oder eine Oberfläche kommt und 
nimmt dem entsprechend von Anfang an den Weg mit der kürzesten Zeit.
Aber woher weiss das Licht, dass sein eingeschlagener Weg der korrekte ist?
Kann man davon ausgehen, dass es die nähere Umgebung anschaut und diese Vergleicht?
Ja, genau davon kann man ausgehen. 
Eine Quantenmechanische Erklärung für das Fermatsche Prinzip ist, 
dass das Licht alle Wege nimmt. Jedoch löschen sich alle anderen Wege aus, 
ausser der Weg mit der kürzester Zeit wird nicht ausgelöscht.
Die genauere Erklärung wie das funktioniert, würde den Rahmen hier sprengen, 
weshalb darauf verzichtet wird.